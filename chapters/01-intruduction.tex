\chapter{Introduction}

\label{Chapter1}

Zero-knowledge proofs are cryptographic techniques that allow the prover to convince the verifier that a statement is true without revealing any information about the statement itself. Their importance lies in enabling ever more secure and private transactions, authentication, and data sharing. Zero-knowledge proofs help maintain privacy and confidentiality while still establishing trust and validity in digital interactions. The need for more efficient and scalable zero-knowledge proofs arises from the growing demand in various applications, including block-chain technology, financial transactions, and data privacy. However, the added properties lead to proof systems, which are computationally expensive and require significant resources. With these concerns in mind a new cryptographic protocol $\plonk$ was created. While there are few articles discussing this protocol formally, there is a lack of middle ground where with clear yet precise explanations. The aim of this article is making the $\plonk$ proof system more accessible. If you are looking for other resources below are a few good starting points:

\begin{enumerate}
    \item \textbf{Articles}
    \begin{enumerate}
        \item \href{https://arxiv.org/pdf/1906.07221.pdf}{How SNARKs work? Definitive Edition}
        \item \href{https://www.di.ens.fr/~nitulesc/files/Survey-SNARKs.pdf}{zk-SNARKs: A Gentle Introduction}
        \item For brave one's: \href{https://eprint.iacr.org/2019/953.pdf}{The PLONK paper}
        \item \href{https://eprint.iacr.org/2022/462.pdf}{Optimization Techniques for PLONK arithmetization}
    \end{enumerate}
    \item \textbf{Blog posts}
    \begin{enumerate}
        \item \href{https://a16zcrypto.com/posts/article/zero-knowledge-canon/#section--1}{Zero Knowledge Canon - Road Map to ZK}
        \item \href{https://vitalik.ca/general/2019/09/22/plonk.html}{Vitalik Buterin's primer on PLONK}
        \item \href{https://research.metastate.dev/plonk-by-hand-part-1/}{PLONK by Hand}
        \item \href{https://blog.lambdaclass.com/all-you-wanted-to-know-about-plonk/}{Lambda class all you need to know about plonk}
        \item \href{https://kobi.one/2021/05/20/plonk-custom-gates.html}{On custom gates}
        \item \href{https://hackmd.io/@jake/plonk-arithmetization}{Deep-dive into arithmetization}
        \item \href{https://medium.com/@VitalikButerin/exploring-elliptic-curve-pairings-c73c1864e627}{Exploring Elliptic Curve Pairings}
        \item \href{https://hackmd.io/@jake/plonk-pcs}{PLONK PCS}
        \item \href{https://hackmd.io/@tazAymRSQCGXTUKkbh1BAg/Sk27liTW9}{MSM optimization}
    \end{enumerate}
    \item \textbf{Lecture}
    \begin{enumerate}
        \item \href{https://zkhack.dev/whiteboard/}{ZK Whiteboard - Lectures on Zero Knowledge cryptography}
    \end{enumerate}
\end{enumerate}

%%%  ===========================================================================
%%%  ===PARTIES INVOLVED========================================================
%%%  ===========================================================================
\section{Parties involved}
In common zero-knowledge protocol there are two parties: one prover $\prover$ and at least one verifier $\verifier$. The objective of the protocol is to convince the verifier that the prover has some secret information. The prover does this by providing an \textit{argument of knowledge}. 


The prover is an entity that possesses a certain witness $w$ and aims to convince the verifier about the knowledge of $w$ without revealing it. Role of the prover is to generate a proof $\pi$, which gets accepted by the verifier if it is true. On the other hand verifier could be multiple entities that, are responsible for checking validity of $\pi$. It is important to mention that verification of the proof is a deterministic algorithm that runs strictly in polynomial time and is orders of magnitude faster than generation of the proof. By looking and the proof the verifier "learns nothing" about $w$. If this sound vague to you, fear not, both entities will be properly formalized.

% \hl{Should I properly define the prover and verifier here or later in this chapter where I have necessary notations defined}

%%%  ===========================================================================
%%%  ====ZERO KNOWLEDGE=========================================================
% ==============================================================================
\section{Zero Knowledge}
The principle of zero-knowledge protocols was raised by the lack of trust to the verifier. What if the prover gives out some addition information contained in the proof that could cause information leakage and be used him. It turns out it is possible prove the statement without giving out any additional information about the secret.  To clarify the protocol we are aiming to construct let me give you a simple example. Consider ordinary deck of cards with 32 red and 32 black. Now I pick one card, let's say a red 8 one and I want to prove to you that the card is red. I can show you the card but that would reveal the information that number of the card has number 8. Instead if I list all of the 32 black cards, you will know that the only remaining are red thus, the card that I have picked needs to be red but you do not know the value of the card. This is the aim of the whole concept. Of course this all happens under the assumption that I as prover am honest and I was not cheating by replacing cards in the standardized pack of cards.  I could have replaced all of the cards with black ones and in this protocol you should not trust me. However PLONK utilizes whole mathematical machinery to make sure that cases like this would not happen and that all of the proof are valid under specific assumptions. By introducing mathematical tool we will be able to prove substantially securely any statement without revealing additional information. At this point it might not be clear why even bother with constructing such proof. Well here are some use-cases that might give you the needed motivation:

\begin{enumerate}
    \item Proving a statement on a private data
    \begin{enumerate}
        \item You want to show to some entity that you have a substantial amount of money on you bank account. For example when taking a loan from a bank that is different from the one where you have you savings.
        \item There is a medical study in which you came to some result and want to show the the result is valid without revealing any of the information about the patients.
    \end{enumerate}
    \item Anonymous authorization:  you want to prove to some website that you have access to a given data or to perform some operation without revealing your IP address or identity
    \item Outsourcing a computation: there is a problem which you need to solve and use some other service to compute it for you. However the solution there is not polynomial algorithm to check that the solution is indeed valid. Then the party doing the computation for you might create a proof that their work was indeed valid.  
\end{enumerate}

plonk requires 5 polynomial commitments with 2 opening proofs...combined degree of polynomials is either $9(m, a)$ smaller proof or $11(m, a)$ larger proofs reduced verifier time for $m$ multiplication gates and $a$ addition gates

Expecting we have gained good motivations let's proceed to necessary background knowledge. All of the definitions in the article will be standard.


%%%  ===========================================================================
%%%  ===ARITHMETIC CIRCUIT======================================================
\section{Arithmetic circuit}
\begin{definition}[Arithmetic circuit]
    $C: \field^n \rightarrow \field$ is a direct acyclic graph with nodes representing arithmetic operations and edges flow of the variables 
\end{definition}

Using circuit it is possible to represent any kind of computation, in fact by using bit operations we can encode an entire program represented in machine code. This is incredibly convenient, since we are able to encode a given computation and compactly proof it's integrity. As we will see later, a circuit defines an n-variate polynomial, with procedure of evaluation. The polynomials could be then bounded by constraint system.

%%%  ===========================================================================
%%%  ===ARGUMENT SYSTEM=========================================================

We will use the ... first introduced by \cite{IOP}. 

\begin{definition}
    An interactive oracle proof system for a relation $\mathcal{R} {0, 1}^* \rightarrow {0, 1}$ is a tuple $(\prover, \verifier)$, where both $\prover, \verifier$ are probabilistic algorithms, that satisfy the following properties:
    \begin{enumerate}
        \item Completeness: $\forall (x, w) \in \mathcal{R}$
    \end{enumerate}
\end{definition}


\section{Argument systems}
Public arithmetic circuit is : $C(x, w) \rightarrow \mathbb{F}_p$ where $x \in \mathbb{F}_p^{n-m}$ is a public statement and $w \in \mathbb{F}_p^m$ is a secret witness. In this construction both the $x, w$ are input parameters to $C$. Prover's goal is to convince the verifier that $\exists w: C(x, w) = 0$. Prover and verifier may interact multiple times until verifier is convinced however in this scenario we will aim towards non-interactive proof. The non-interactivity system has a setup procedure: $S(C) \rightarrow (S_p, S_v)$ which are public parameters of prover and verifier. The whole system is made up of tree algorithms:
\begin{enumerate}
    \item Setup: $S(C) \rightarrow (S_p, S_v)$
    \item Proof: $P(S_p, x, w) \rightarrow \pi$
    \item Verification: $V(S_v, x, \pi) \rightarrow T/F$
\end{enumerate}

Setup produces common reference string 
algorithms are all probabilistic non-uniform polynomial time algorithms.
the verifier takes and outputs 1 if and only if the proof is accepted


\subsection{Properties of an arguments system}
Before going any further is it important to discuss that the proof system $(S, P, V)$ has to be \textit{complete, knowledge sound} and additionally  \textit{zero-knowledge}.

Let $R$ be an efficiently computable binary relation on binary strings, with pairs $(x, w) \in R$ consisting of $x$ statement and $w$ witness. Then $L$ is the language consisting of statements in $R$. 

\begin{definition}[Perfect Completeness]
    A proof system is complete if for all adversaries $\mathcal{A}$: $Pr[(S_v, S_p) \leftarrow Setup(1^k); \pi \leftarrow Prove(S_p, x, w): Verify(S_v, x, \pi) = 1 | (x, w) \in R] = 1$.
\end{definition}

If the $P$ knows the $w$, then the $V$ accepts.

\begin{definition}[Perfect Completeness]
    A proof system if perfectly sound if for all polynomial size families $\{x_n\}$ of statements $x_n \notin L$ and all adversaries $\mathcal{A}$
    $$Pr[(S_v, S_p) \leftarrow Setup(1^k); \pi \leftarrow Prove(S_p, x_n): Verify(S_v, x, \pi) = 1] = 0 $$

    A computation knowledge soundness could be denoted as
    $$Pr[(S_v, S_p) \leftarrow Setup(1^k); \pi \leftarrow Prove(S_p, x_n): Verify(S_v, x, \pi) = 1] < \epsilon(n) $$
\end{definition}

\begin{definition}[Zero-Knowledge]
    $(Setup, Prove, Verify)$ is zero-knowledge for a circuit $C$ if there is an efficient simulator $Sim$ such that $\forall x \in \mathbb{F}^n$ such that $\exists w: C(x, w) = 0$ the distribution 
    $(C, S_p, S_v, x, \pi); (S_p, S_v) \leftarrow S(C), \pi \leftarrow Prove(S_p, x, w )$ i
    is indistinguishable from the distribution: 
    $(C, S_p, S_v, x, \pi) ; (S_p, S_v, \pi) \leftarrow Sim$
\end{definition}

General intuition behind zero-knowledge is that the communication between the verifier and prover reveals nothing about the secret. This is formalized by existence of simulator not knowing the secret the can falsify the transcript of communication. Be cautious because there are multiple different definition of the zero knowledge property. % \hl{briefly explain, thaler page 170}. The definition defined above is honest verifier zero knowledge HVZK that expect the verifier behaves according to the specified protocol. The difference from basic zero-knowledge is that in normal zk the verifier can be malicious. PLONK expects HVZK. 

In a nutshell, completeness says that if the $P$ has information than $V$ accepts, knowledge soundness that if the $P$ does not have the information the $V$ does not accept. Moreover the proof system is zero-knowledge if all the information that $V$ has access to $(C, S_p, S_v, x, \pi)$ reveal nothing about $w$.

Succintness in terms of plonk paper:
\begin{itemize}
    \item The preprocessing1 phase/SRS generation run time is quasilinear in circuit size.
    \item The prover run time is quasilinear in circuit size.
    \item The proof length is logarithmic2 in circuit size.
    \item he verifier run time is polylogarithmic in circuit size.
\end{itemize}

where quasilinear (log-linear) time is: $O(n \log{n}^k)$ for positive constant $k$ and polylogarithmic time is $O(\log{n}^k)$ for constant $k$


%%%  ===========================================================================
%%%  ===INTERACTIVITY===========================================================
\section{Interactive and Non-interactive protocols}
Many of the basic examples of zero-knowledge algorithms go in the way, verifier queries the prover with random values until he has gained a sufficient certainty. With each iteration the probability of prover guessing the answer diminished. While the examples serve good for explaining the concept in practice they have many disadvantages. In most cases we want the protocol to be used in some communication via internet where queries take dramatically longer than interacting locally with some oracle. Moreover interactivity, make it hard to use the protocol in multi-party setup, because now the prover must answer repeatedly to all of the verifier. In non-interactive setup in would be sufficient to send just one proof to each verifier which would convince them. Since the communication protolcol is non-interactive we will be aiming towards this setting also show how it is possible to convert a protocol from interactive to non-interactive. 

%%%  ===========================================================================
%%%  ===CONTEXT=================================================================
\section{PLONK in context of other proof systems}
In the landscape of cryptographic protocols, several notable constructions have emerged, each contributing unique features. Groth16, Marlin, Bulletproofs, STARKs and DARK are among the distinguished protocols that have got attention for their advancements in privacy-preserving computations. These protocols demonstrate ongoing quest for efficient, secure, and scalable solutions. Each of them has some specifics and make certain tradeoff the verification time and size of the proof and setup parameters. In figure 2.1 you can see the differences for yoursef the details would be covered in the chapter X.
